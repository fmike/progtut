%% LaTeX-Beamer template for KIT design
%% by Erik Burger, Christian Hammer
%% title picture by Klaus Krogmann
%%
%% version 2.1
%%
%% mostly compatible to KIT corporate design v2.0
%% http://intranet.kit.edu/gestaltungsrichtlinien.php
%%
%% Problems, bugs and comments to
%% burger@kit.edu

\documentclass[18pt]{beamer}

%% SLIDE FORMAT

% use 'beamerthemekit' for standard 4:3 ratio
% for widescreen slides (16:9), use 'beamerthemekitwide'

\usepackage{templates/beamerthemekit}
% \usepackage{templates/beamerthemekitwide}

%% TITLE PICTURE

% if a custom picture is to be used on the title page, copy it into the 'logos'
% directory, in the line below, replace 'mypicture' with the 
% filename (without extension) and uncomment the following line
% (picture proportions: 63 : 20 for standard, 169 : 40 for wide
% *.eps format if you use latex+dvips+ps2pdf, 
% *.jpg/*.png/*.pdf if you use pdflatex)

%\titleimage{mypicture}

%% TITLE LOGO

% for a custom logo on the front page, copy your file into the 'logos'
% directory, insert the filename in the line below and uncomment it

%\titlelogo{mylogo}

% (*.eps format if you use latex+dvips+ps2pdf,
% *.jpg/*.png/*.pdf if you use pdflatex)

%% TikZ INTEGRATION

% use these packages for PCM symbols and UML classes
% \usepackage{templates/tikzkit}
% \usepackage{templates/tikzuml}

% the presentation starts here
\usepackage{graphicx}
\usepackage{listings}
\lstdefinelanguage{Pseudo}
{keywords={Function,return,assert,while,for,to,if,else,end,do,Procedure},%
emph={FALSE,TRUE},
emphstyle=\color{colconst},
sensitive=true,%
comment=[l]{//},%
string=[b]",%
}
\lstset{basicstyle=\ttfamily}

\usepackage{stmaryrd}
\usepackage[utf8]{inputenc}

\title[Prog Tut Nr. 1]{Tutorium Programmieren}
\subtitle{Tut Nr.1: Kennenlernen, Organisatorisches, Einführung Java}
\author{Julian Geppert}
\date{29.10.2013}

\institute{Institut f\"ur theoretische Informatik}
% Bibliography

\usepackage[citestyle=authoryear,bibstyle=numeric,hyperref,backend=biber]{biblatex}
\addbibresource{templates/example.bib}
\bibhang1em

\begin{document}

% change the following line to "ngerman" for German style date and logos
\selectlanguage{ngerman}

%title page
\begin{frame}
	\titlepage
\end{frame}

%table of contents
\begin{frame}{Outline/Gliederung}
	\tableofcontents
\end{frame}

\section{Kennenlernen}
\subsection{Meine Wenigkeit}
\begin{frame}{Vorstellung / Kennenlernen}
\begin{block}{Meine Wenigkeit}
	\begin{itemize}
		\item Name: Julian Geppert
		\item Studiengang: Informatik Bachelor
		\item E-Mail: \url{julian.geppert@student.kit.edu}
		\item Programmiererfahrung: Java, C, C++, PHP, (Ruby)
	\end{itemize}
\end{block}
\end{frame}

\subsection{Vorstellungsrunde}
\begin{frame}{kurze Vorstellungsrunde}
\begin{block}{Und wer seid ihr so?}
	\begin{itemize}
		\item Name, Studiengang, Programmiererfahrung?
		\item Erwartungen an Vorlesung / Tutorium?
	\end{itemize}
	\end{block}
	\begin{figure}%
	\center
	\includegraphics[width=0.4\columnwidth]{bild.jpg}%
	\end{figure}
\end{frame}

\section{Organisatorisches}
\subsection{Das Tutorium}
\begin{frame}{Das Tutorium}
\begin{block}{Termin}
	Jeden Dienstag um 9:45-11:15 Uhr in Raum -109 Geb. 50.34
\end{block}
\begin{block}{Kollaboration}
	Folien und Vorbereitung in Zusammenarbeit mit
	\begin{itemize}
		\item Ingo Sobik, Tutorium Di. 11:30-13:00 Uhr
		\item Michael Friedrich, Tutorium Di \& Do 11:30-13:00 Uhr
	\end{itemize}
\end{block}
\begin{block}{Was machen wir hier?}
	\begin{itemize}
		\item Vorlesungsstoff besprechen
		\item Übungsblätter besprechen
		\item organisatorische und natürlich fachliche Fragen klären
		\item gemeinsam Programmieren mit Java lernen
	\end{itemize}
\end{block}
\end{frame}

\subsection{Übungsschein}
\begin{frame}{Übungsschein}
	\begin{block}{Übungsblätter}
		\begin{itemize}
			\item circa alle 2 Wochen neues Übungsblatt (siehe VL-Website)
			\item insgesamt 6 Übungsblätter
		\end{itemize}
	\end{block}
	\pause
	\begin{block}{Übungsschein}
		\begin{itemize}
			\item $\geq 50\%$ der Punkte sind zu erreichen
			\item Übungsschein ist Pflicht um zu den beiden Abschlussaufgaben zugelassen zu werden
			\item Übungsaufgaben sind wichtige Vorbereitung auf die Prüfung
		\end{itemize}
	\end{block}
\end{frame}


\begin{frame}{Übungsschein}
	\begin{alertblock}{Warnung!}
		\begin{itemize}
			\item \textbf{Nicht abschreiben!}
			\item Bei nachgewiesenem Abschreiben erhält man \textbf{keinen Übungsschein!}
			\item Somit ist man nicht zur Prüfung zugelassen!
			\item Programmieren ist Teil der Orientierungsprüfung!
		\end{itemize}
	\end{alertblock}	
\end{frame}

\begin{frame}{Übungsschein}
	\begin{block}{erstes Übungsblatt}
		\begin{itemize}
			\item \dots bereits auf Vorlesungswebsite verfügbar
			\item Abgabezeitraum: 04.11. 13 Uhr bis 11.11. 13 Uhr
		\end{itemize}
	\end{block}
\end{frame}

\begin{frame}{Nützliche Links}
	\begin{block}{Links}
	\begin{itemize}
		\item Vorlesungswebsite
		\item[] \url{baldur.iti.kit.edu/programmieren}
		\item Praktomat (Abgabe der Übungsblätter)
		\item[] \url{praktomat.info.uni-karlsruhe.de/}
		\item Ilias
		\item[] \url{ilias.studium.kit.edu}
		\item Studierendenportal (Anmeldung Übungsschein/Prüfung)
		\item[] \url{studium.kit.edu}
	\end{itemize}
	\end{block}
\end{frame}

\begin{frame}{Ilias}
	\begin{block}{Ilias?}
	\begin{itemize}
		\item Ilias = eLearning System des KIT
		\item Zu erreichen unter \url{ilias.studium.kit.edu}
	\end{itemize}
	\end{block}
	\pause
	\begin{block}{Inhalte}
	\begin{itemize}
		\item Diskussionsforum zur Vorlesung
		\item Allgemeines Forum für Fragen
		\item Foren für jedes Tutorium
		\item Programmieren-Wiki
		\item \textcolor[rgb]{1,0,0}{Bitte keine Lösungen posten!}
	\end{itemize}
	\end{block}
	Passwort für Programmieren: \hspace{1cm} \url{prog-13}
\end{frame}

\begin{frame}{Erinnerungen / Deadlines}
	\begin{itemize}
		\item Abgabe des Disclaimers bis spätestens 4.11.2013
		\item Anmeldung beim Praktomaten bis spätestens 4.11.2013
		\item Anmeldung zum Übungsschein bis spätestens 23.12.2013	
		\item danach Anmeldung zu den Abschlussaufgaben
	\end{itemize}		
\end{frame}

\section{Java Installation / Einführung}
\begin{frame}
\end{frame}

\appendix
\beginbackup

%\begin{frame}[allowframebreaks]{References}
%	\printbibliography
%\end{frame}

\backupend

\end{document}
